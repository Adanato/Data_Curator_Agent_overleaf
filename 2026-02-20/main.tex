\meetingheader
    % Date
    {Feb 20, 2026}
    % Git
    {commit\_id}
    % Links
    {\meetinglinks{
        \meetinglink{Feiyang - Tasks Research Paper Baselines}{https://docs.google.com/document/d/1zWGD77TwfEV1YIu0qf12wiF8QwLmFGJvXv5zDWpY8DY/edit?tab=t.cptniv3ucalr\#heading=h.by11bqpevc2}
	\meetinglink{Feiyang - Agent Skills}{https://docs.google.com/document/d/12R-l8-FlARc7q0jRI8gKOMHPRlXV-e5oCmwdQANgV9w/edit?tab=t.0\#heading=h.p4acw1gohv8l}
        \meetinglink{Hanze - Agent Design Google Doc}{https://docs.google.com/document/d/1W9h013IDs60LMBDfKyyoPzRYBxbdadayO2ePJW5Eo4U/edit?tab=t.0}
        \meetinglink{Ruoxi - Benchmark Design}{https://docs.google.com/document/d/16LiPuHChJ2uD9LK5LwuYoFj7qshdletW5NUXXVaNh74/edit?usp=sharing}
    }}
    % Summary
    {\begin{itemize}
        \item Read documentation on autogen
	\item tested execution boundaries: docker, local
	\item bothered feiyang some more
    \end{itemize}}
    % Housekeeping
    {Housekeeping items (e.g. minor updates)...}



\newpage
\section{Single Agent System}

Motivation: Simplicity of testing and observation of failure modes and patterns to motivate Multi Agent System (MAS)




\begin{figure}[ht]
\centering
\begin{forest}
  for tree={
    font=\ttfamily,
    grow'=0,
    child anchor=west,
    parent anchor=south,
    anchor=west,
    calign=first,
    edge path={
      \noexpand\path [draw, meetingblue]
        (!u.south west) +(7.5pt,0) |- node[fill,inner sep=1.25pt] {} (.child anchor)\forestoption{edge label};
    },
    before typesetting nodes={
      if n=1 {insert before={[,phantom]}} {}
    },
    fit=band,
    before computing xy={l=15pt},
  }
  [\textcolor{meetingblue}{\textbf{data-curator-agent/}}
    [\textcolor{meetingblue}{src/}
      [agents/
        [curator\_agent.py]
      ]
      [tools/
        [\_\_init\_\_.py]
      ]
      [main.py]
    ]
    [\textcolor{meetingblue}{tests/}
      [test\_agent.py]
    ]
    [pyproject.toml]
    [.envrc]
    [README.md]
  ]
\end{forest}
\caption{Data Curator Agent repository structure}
\label{fig:repo-structure}
\end{figure}


\newpage
\subsection{Tool Ingestion}

Tools are plain Python functions with typed signatures. Each is wrapped in AutoGen's
\texttt{FunctionTool}, which introspects the function signature and docstring to generate
an OpenAI-compatible JSON tool schema:

\begin{verbatim}
def profile_datasets(
    sample_n: Annotated[int,
        "Number of rows to sample (0 = all)"] = 0,
    seed: Annotated[int, "Random seed"] = 42,
) -> str:
    """Load all candidate datasets and compute
    quality profiles."""
    ...
\end{verbatim}

\noindent becomes:

\begin{verbatim}
{
  "name": "profile_datasets",
  "description": "Load and profile ALL candidate ...",
  "parameters": {
    "type": "object",
    "properties": {
      "sample_n": {
        "type": "integer",
        "description": "Number of rows to sample ..."
      },
      "seed": {"type": "integer", ...}
    }
  }
}
\end{verbatim}

\noindent The full tool list is passed on \textbf{every} LLM call via
\texttt{client.create(messages, tools=self.\_tools)}.
The model sees all available tools at every iteration and decides whether to call one
or return text.



\newpage
\subsection{Conversation History}

The agent maintains a linear message history that grows with each iteration.
The pseudocode below shows the exact message types appended at each step:

\begin{verbatim}
messages = [
  SystemMessage(system_prompt),         # role + task + workflow
  UserMessage(task_description),        # "Profile FineVision5..."

  # --- iteration 1: ACT ---
  AssistantMessage(content=[            # model chooses a tool
    FunctionCall(
      name="profile_datasets",
      arguments='{"sample_n": 0}')
  ]),
  # --- iteration 1: OBSERVE ---
  FunctionExecutionResultMessage([      # tool output fed back
    FunctionExecutionResult(
      name="profile_datasets",
      content='{"num_datasets":5, ...}')
  ]),

  # --- iteration 2: ACT ---
  AssistantMessage(content=[
    FunctionCall(
      name="compute_mix_ratio",
      arguments='{"quality_scores":"...", ...}')
  ]),
  # --- iteration 2: OBSERVE ---
  FunctionExecutionResultMessage([
    FunctionExecutionResult(
      name="compute_mix_ratio",
      content='{"recipe": {...}}')
  ]),

  # --- iteration 3: TERMINATE ---
  AssistantMessage(content="Final summary: ...")
  # text response -> loop breaks
]
\end{verbatim}

\noindent \textbf{Termination:} The loop ends when the model returns a plain text string
instead of a tool call. The agent sets \texttt{stop\_reason = "text\_response"} and breaks.
Other exit conditions: repeated tool calls ($\geq 3\times$) or \texttt{max\_iterations}.

\newpage
\subsection{System Prompt (Preselection)}

The system prompt is assembled from composable sections selected by \texttt{prompt\_style}
in the task YAML. Below is the preselection style used in experiments:

\begin{verbatim}
You are a data curation agent performing the
**preselection phase**. Your job is to profile
candidate datasets, assess their quality, and compute
an optimal mix ratio for fine-tuning.

## Task
Profile the FineVision5 candidate datasets, compute
quality scores from existing metadata, and determine
an optimal mix ratio for a 100k-sample VLM training
mixture.

## Target Model
Qwen/Qwen2.5-VL-7B-Instruct

## Sample Budget
100000 total samples

## Available Datasets
LLaVA_Instruct_150K, vqav2, ocrvqa, captcha,
CoSyn_400k_chart

## Preselection Workflow
Follow these three steps in order:

### Step 1: Profile
Call profile_datasets to load all candidate datasets
and compute quality scores from their metadata.

### Step 2: Assess
Analyze the quality profiles. Consider dataset sizes,
quality score distributions, and fitness for the
target model.

### Step 3: Compute Mix
Call compute_mix_ratio with quality_scores and
effective_sizes to determine the optimal mix ratio.
Total samples must fit within the sample budget.
\end{verbatim}





\meetingtasks{
    \item \todo{Evaluate Single Agent System Via Baselines}
    \item \todo{Locate failure modes that motivate greater Multi Agent System Design. What motivatees a dedicated agent for each part? What would such agents solve that a single strong reasoning agent couldn't?}
}
